\chapter{Style guidelines}
\label{section:style}

\section{Object oriented design}

The ExtJS toolkit was founded on a solid OO-inspired design, using classes and inheritance. The application
is developed using the same style as ExtJS, so that our code integrates well with existing ExtJS components.

To declare a new class, one usually starts with defining the constructor. It is customary, but not required, 
to have a 'configuration' object as the first parameter. Configuration parameters are object/value pairs 
that configure specific parts of an object's instantiation, avoiding large sparse parameter lists. The 
constructor calls the constructor of its parent manually (as there is no way of doing this automatically). 
Listing \ref{listing:declaringclass} shows an example of this process. 

\begin{lstlisting}[caption={Declaring a new class}, label=listing:declaringclass]
// Zarafa.ui.mail.PreviewPanel constructor
Zarafa.ui.mail.PreviewPanel = function(config)
{
	// call the parent constructor
	Zarafa.ui.mail.PreviewPanel.superclass.constructor.call(this, config);
};

// Zarafa.ui.mail.PreviewPanel extends Ext.Panel
Ext.extend(Zarafa.ui.mail.PreviewPanel, Ext.Panel);
\end{lstlisting}

The call to {\tt Ext.extend()} expresses that {\tt Zarafa.ui.mail.PreviewPanel} extends (is a child class of) 
{\tt Ext.Panel}. It is now possible to substitute an instance of the former for the latter. If the class 
you're writing is not a child class of anything, simply extend from {\tt Object}.
ExtJS offers a range of handy tools besides {\tt Ext.extend()} that help us write modular code.

\section{Architectural style}

\subsection{Namespaces}

The ExtJS toolkit provides the {\tt Ext.namespace} function that declares a namespace. We place chunks of
related functionality in namespaces and let the code tree reflect the namespace hierarchy. For example, classes
in the {\tt Zarafa.ui.mail} namespace are defined in Javascript files that can be found in {\tt zarafa/ui/mail}.

\subsection{Singletons}

Static functions can be declared inside an \emph{singleton} object. A singleton, in OO design, is a class which
only has a single instance. This can be easily simulated in Javascript. For example, consider Listing 
\ref{listing:singleton}.

\begin{lstlisting}[caption={Declaring a singleton.}, label=listing:singleton]
/**
 * @class Zarafa.comm.Mail
 * 
 * The Mail object is a singleton containing several convenience methods for working with email messages.
 * 
 * @singleton
 */
Zarafa.comm.Mail = 
{
	openMail : function(storeId, entryId, callBack, errorCallback)
	{
		// implementation goes here
	}
};
\end{lstlisting}

This creates a function {\tt openMail} in the {\tt Zarafa.comm.Mail} object, which can be called anywhere in the
code as {\tt Zarafa.comm.Mail.openMail()}. By putting static functions in a singleton, we can group such
functions and document them as a unit.

\subsection{Enums}

An enumeration, or \emph{enum} for short is an object that contains a set of constants. They can be 
implemented as a JavaScript object.

\begin{lstlisting}[caption={Declaring an enumeration}, label=listing:enum]
/**
 * @class Zarafa.comm.Appointment.BusyStatus
 * 
 * Enumerates the different busy status types. 
 * 
 * @singleton
 */
Zarafa.comm.Appointment.BusyStatus = 
{
	
	// property documentation omitted for clarity
	free : 0,
	tentative : 1,
	busy : 2,
	outOfOffice : 3
};
\end{lstlisting}

Listing \ref{listing:enum} shows the {\tt Zarafa.comm.Appointment.BusyStatus} enum, enumerating all possible
values the 'busy status' value an appointment may take.  Enums should be used to avoid so called 'magic numbers'
in code. 

\section{Code style}

To keep our sanity of ourselves and the people working on this project after us it's important to write code
that is both clear, simple, and uniform in style. 

\subsection{Naming}

Copying the convention from ExtJS, namespaces are all in lowercaps except for the Z in 'Zarafa' 
(ie {\tt Zarafa.ui.mail}). Classes are camelcase starting with uppercase ({\tt Zarafa.ui.mail.MailGridPanel})
and variables, methods, and fields are camelcase starting with lowercase ({\tt Zarafa.comm.Mail.openMail()}).

\subsection{Method length}

Generally when methods span many lines it is possible to break up the function is maller pieces. This increases
readability as the reader does not need to know exactly how each part of the function is implemented to understand
the general idea. Break up functions of more than 30-50 lines or so should be broken up into smaller ones if 
possible. 

\section{Documentation}
	
I would like to stress that any piece of code that is not documented \emph{is not done}. Please don't close
tickets on uncommented code. We use the {\tt ext-doc} documentation tool provided by the ExtJS people.

This documentation tool allows documentation of Javascript code much like Javadoc or Doxygen. This section
describes how to document your code.
Since Javascript is a very dynamic language it's pretty much impossible to detect class, method, and field
definitions, and the relationships between them. Therefore documenation is quite explicit and you have to
declare classes, methods, and fields manually. This section briefly describes the most important aspects
of documenting with ext-doc. Please refer to the ext-doc documentation wiki for more information.

\subsection{Documenting classes}

You declare a class using the {\tt @class} statement inside a {\tt /** */} multi-line comment. You can
use {\tt @extends} to indicate that the class is a subclass of another class. After these two statements
you can fill in a description of the class. Optionally you can use the {\tt @singleton} statement to declare 
the class a singleton. 
The constructor can be documented separately and is documented much like a method. Finally use {\tt @cfg}
to declare configuration options for this class. Note that parameters and configuration options are typed.
A full example is shown in Listing \ref{listing:classdoc}.

\begin{lstlisting}[caption={Declaring a new class}, label=listing:classdoc]
/**
 * @class Zarafa.ui.ContextContainer
 * @extends Ext.Container
 * [ Description goes here ]
 * @constructor
 * @parameter config configuration object
 * @cfg String name the name of the context container.
 * @cfg Ext.Component defaultItem a default component to be shown if a context does 
 * not provide a component for this ContextContainer. 
 */
Zarafa.ui.ContextContainer = function(config)
{
	Zarafa.ui.ContextContainer.superclass.constructor.call(this, config);
};
\end{lstlisting}

\subsection{Documenting methods}

Listing \ref{listing:methoddoc} shows how to document methods using the {\tt @method} tag. Parameters can be
specified using {\tt @param} with a type, name, and description. If an argument is optional please make
this explicit by using {\tt (optional)} as exemplified by the errorCallBack parameter in listing 
\ref{listing:methoddoc}.

\begin{lstlisting}[caption={Declaring a new class}, label=listing:methoddoc]
Zarafa.comm.Mail = 
{

	/**
	 * Opens a mail asynchronously.
	 * 
	 * @param {String} storeId a MAPI ID that corresponds to a MAPI store 
	 * @param {String} entryId a MAPI ID that corresponds to an item in the given store
	 * @param {Function} callback a function that will be called when the message has arrived 
	 *	from the server  
	 * @param {Function} errorCallback (optional) a function that will be called when an error 
	 *	occurred retrieving the mail
	 * @method openMail
	 */
	openMail : function(storeId, entryId, callBack, errorCallback)
	{
		// implementation
	}
	

};
\end{lstlisting}

\subsection{Documenting insertion points}
\label{section:docinsert}


